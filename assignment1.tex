\documentclass{article}
\usepackage{amsmath}
\usepackage{enumitem}
\usepackage{amsfonts}
\providecommand{\mydet}[1]{\ensuremath{\begin{vmatrix}#1\end{vmatrix}}}
\providecommand{\myvec}[1]{\ensuremath{\begin{bmatrix}#1\end{bmatrix}}}
\providecommand{\cbrak}[1]{\ensuremath{\left\{#1\right\}}}
\providecommand{\brak}[1]{\ensuremath{\left(#1\right)}}
\title{ASSIGNMENT-1}
\date{\today}
\begin{document}
\maketitle{Questions}
\begin{enumerate}
		\section{VECTOR GRAPHS}
\item The position vectors of points $P$ and $Q$ are $\vec{p}$ and $\vec{q}$ respectively. The point $R$ divides linesegment $PQ$ in the ratio $3 \colon 1$ and $S$ is the mid-point of line segment $PR$. The position vector of $S$ is :
        \begin{enumerate}[label=(\Alph*)]
                \item $\frac{\vec{p} + 3\vec{q}}{4}$
                \item $\frac{\vec{p} + 3\vec{q}}{8}$
                \item $\frac{5\vec{p} + 3\vec{q}}{4}$
                \item $\frac{5\vec{p} + 3\vec{q}}{8}$
        \end{enumerate}

                                \item The angle which the line $\frac{x}{1} = \frac{y}{-1} = \frac{z}{0} $ makes with the positive direction of Y-axis is :
                                        \begin{enumerate}[label=(\Alph*)]
                                                \item $\frac{5\pi}{6}$
                                                \item $\frac{3\pi}{4}$
                                                \item $\frac{5\pi}{4}$
                                                \item $\frac{7\pi}{4}$
                                        \end{enumerate}

                                \item The Cartesian equation of the line passing through the point $(1,-3,2)$ and parallel to the line\\
                                        $\vec{r} = (2+\lambda)\hat{i} + \lambda \hat{j} + (2\lambda-1)\hat{k}$ is
\begin{enumerate}[label=(\Alph*)]
        \item \(\frac{x-1}{2} = \frac{y+3}{0} = \frac{z-2}{-1}\)
        \item \(\frac{x+1}{2} = \frac{y-3}{1} = \frac{z+2}{2}\)
        \item \(\frac{x+1}{2} = \frac{y-3}{0} = \frac{z+2}{-1}\)
    \item \(\frac{x-1}{1} = \frac{y+3}{1} = \frac{z-2}{2}\)
\end{enumerate}
\end{enumerate}
\begin{enumerate}
		\section{INTEGRALS}
\item $\int \frac{1}{x(\log x)^2} dx$ is equal to :                                                                                                                                      \begin{enumerate}[label=(\Alph*)]
\item $2log(logx)+c$
\item $-\frac{1}{logx}+c$
\item $\frac{(logx)^3}{3}+c$
\item $\frac{3}{(logx)^3}+c$
        \end{enumerate}

        \item The value of $\int_{-1}^{1} x|x|dx$ is :
                \begin{enumerate}[label=(\Alph*)]
\item $\frac{1}{6}$
\item $\frac{1}{3}$
\item $\frac{1}{6}$
\item $0$
        \end{enumerate}
\end{enumerate}
\begin{enumerate}
		\section{AREA AND CURVES}
\item Area of the region bounded by curve $y^2 = 4x$ and the $X-axis$ between $x = 0$ and $x = 1$ is :
        \begin{enumerate}[label=(\Alph*)]
\item $\frac{2}{3}$
\item $\frac{8}{3}$
\item $3$
\item $\frac{4}{3}$
\end{enumerate}
                \item Given a curve $y = 7x - x^3$ and $x$ increases at the rate of $2 units per second$. The rate at which the slope of the curve is changing, when $x = 5$ is :
                \begin{enumerate}[label=(\Alph*)]
                \item $-60 units/sec$                                                                                                                                                            \item $60 units/sec$
                \item $-70 units/sec$
                \item $-140 units/sec$
                \end{enumerate}
\end{enumerate}
		\begin{enumerate}
			\section{MATRIX}
		\item If $A$ = $\myvec{9 & c & -1 \\b & 0 & 5\\1 & -5 & 5}$ is a skew symmetric matrix, then the value of $2a - (b+c)$ is :                                                   \begin{enumerate}[label=(\Alph*)]
        \item ${0}$
        \item ${1}$
        \item ${-10}$
        \item ${10}$
        \end{enumerate}

\item If $A$ is a square matrix of order $3$ such that the value of $|adj.A| = 8$,then the value of $|A^T|$ is :                                                                      \begin{enumerate}[label=(\Alph*)]
                \item $\sqrt{2}$
                        \item $-\sqrt{2}$
                        \item $8$
                        \item $2 \sqrt{2}$
        \end{enumerate}


        \item If inverse of matrix $\myvec{7 & -3 & -3 \\-1 & 1 & 0\\-1 & 0 & 1}$ is the matrix $\myvec{1 & 3 & 3\\1 & \lambda & 3\\1 & 3 & 4}$ ,  then value of $\lambda$ is :                                                                                          \begin{enumerate}[label=(\Alph*)]
                \item $-4$
                        \item $1$
                        \item$3$                                                                           \item $4$                                                          \end{enumerate}


        \item If $\myvec{x & 2 & 0} \myvec {5 \\ -1 \\ x}$ = $ \myvec{3 & 1} \myvec {-2 \\ x}$, then value of $x$ is :
                \begin{enumerate}[label=(\Alph*)]
                \item $-1$
                        \item $0$
                        \item $1$
                        \item $2$
\end{enumerate}
	\item Find the matrix $A ^{2}$ , where $A$ = [$a_{ij}$] is a $2\times2$ matrix whose elements are given by $a_{ij}$ = maximum $(i,j)$ - minimum $(i,j)$ :
                \begin{enumerate}[label=(\Alph*)]
                \item $\myvec{0 & 0\\0 & 0}$
                        \item $\myvec{0 & 1\\1 & 0}$
                                \item $\myvec{1 & 0 \\0 & 1}$
                                        \item $\myvec{1 & 1\\0 & 1}$
		\end{enumerate}
		\end{enumerate}
				\begin{enumerate}
						\section{PROBABILITY}
\item If A and B are events such that$ P(A/B) = P(B/A) \neq 0$, then
:
                \begin{enumerate}[label=(\Alph*)]
                                                                                                                                                                                 \item $A \subset B, but A \neq B$
                \item $A = B$
                \item $A\cap B = \phi$
                \item $P(A) = P(B)$                                                                                                                                                              \end{enumerate}
				\end{enumerate}

				\begin{enumerate}
						\section{FUNCTIONS}
        \item A function $f :\mathbb{R} \rightarrow \mathbb{R}$ defined as $f(x) = x^2 - 4x +5 $ is :
                \begin{enumerate}[label=(\Alph*)]
                \item injective but not surjective
\item surjective but not injective
\item both injective and surjective
\item neither injective nor surjective
\end{enumerate}
\item The function $f(x)$=$\frac{x}{2}$+$\frac{2}{x}$has a local minima at x is equal to:

        \begin{enumerate}[label=(\Alph*)]
                        \item$2$
                        \item$1$
                        \item$0$
                        \item$-2$
                \end{enumerate}
				\end{enumerate}

				\begin{enumerate}
						\section{DIFFERENTIAL EQUATIONS}
						    \item If $xe^{y}=1$,then the value of $\frac{dy}{dx}$ at $x = 1$ is :
                \begin{enumerate}[label=(\Alph*)]
                        \item $-1$
                        \item $1$
                        \item $-e$
                        \item $-\frac{1}{e}$
                \end{enumerate}
                        \item Derivative of $e^{sin^2x}$ with respect to cos x is :
                                \begin{enumerate}[label=(\Alph*)]
                                        \item $\sin x  e^{\sin^2x}$
                                        \item $\cos x  e^{\sin^2x}$
                                        \item $-2\cos x e^{\sin^2x}$
                                        \item $-2\sin^2 x cos x  e^{\sin^2x}$
                                \end{enumerate}
                        \item The order of the differential equation $\frac{d^4y}{dx^4} - \sin[\frac{d^2y}{dx^2}]=5$ is
                                \begin{enumerate}[label=(\Alph*)]

                                        \item $4$
                                        \item $3$
                                        \item $2$
                                        \item notdefined
				\end{enumerate}
				\end{enumerate}
				
						\begin{enumerate}
						\section{ASSERTION}
\item Assertion-Reason Based Questions

Direction: In questions numbers $19$ and $20$, two statements are given one labelled Assertion (A) and the other labelled Reason (R). Select the correct answer from the following options:
                \begin{enumerate}
\item Both Assertion (A) and Reason (R) are true and the Reason (R) is the correct explanation of the Assertion (A).
\item Both Assertion (A) and Reason (R) are true and Reason (R) is not the correct explanation of the Assertion (A).
\item Assertion (A) is true, but Reason (R) is false.
\item Assertion (A) is false, but Reason (R) is true.
                \end{enumerate}
        Assertion (A): Domain of $y$ = $\cos^{-1}x$ is $[-1, 1]$.

                Reason (R) : The range of the principal value branch of $y$ = $\cos^{-1}x$ is $[0, \pi]-{\frac{\pi}{2}}$.
\end{enumerate}
\end{document}
